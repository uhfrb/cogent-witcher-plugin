\defineability{blinding}{Blinding}
{
    All combattants (with vision) within sight must roll a REF-check, CL2. Every level they fail their check by
    results in one turn of \hyperref[modifier:blindness]{blindness}. If the CL is met exactly, the shielding of one's eyes results in a penalty
    of -1D6 for this turn. If the CL is exceeded, no adverse effects occur.
}

\defineability{freezingfog}{Freezing Fog}
{
    The user throws a cloud of fog and tiny ice-shards in every direction.
    Every combattant with exposed skin takes level one injuries according to
    the amount of skin in contact with the attack. If players want to dive for
    cover, that counts as a defensive action and is pitted against the user's roll.
}

\defineability{restoration:spectre}{Restoration (Spectres)}
{
    The spectre turns to fog or smoke and creates a number of copies in the near vicinity. These
    copies are harmless and can be immediately defeated by any victory level, but for any $n$ copies
    alive at the end of one round of combat, the spectre may heal one leven $n$ injury. The copies
    roll half the original's dice, but penalties to the original do not apply. The original is forced
    to reappear once the last copy is taken out.
}

\defineability{selfdetonation:rotfiend}{Self-detonation (Rotfiends)}
{
    The rotfiend begins to swell and wriggle. After one turn of combat, any combattant standing next to
    the creature without cover receives an injury (greater the closer) and is poisoned. If they are not
    treated within a day, grave illness and even death may ensue. This ability is always triggered once
    the injuries dealt to the Rotfiend reduce its combat roll to a third of its original or if it receives
    a level 4 injury.
}

\defineability{fly:humanoid}{Fly}
{
    The creature can fly for several minutes or even hours at a time. While flying, any melee attack gains \textit{charge}, 
    however, it takes at least one turn of combat to circle around for the next swoop. If any injury is inflicted upon a 
    flying creature, it is staggered or tripped,
    it will fall for at least one turn of combat. If it reaches the ground in that time, it will take damage as appropriate,
    but certainly receive the \textit{prone} combat-modifier. To take off, the creature must not receive any damage
    or attack for one turn of combat, this may be attempted in a defensive action.
}